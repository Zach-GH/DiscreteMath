\section*{Problem 1}

Indicate whether the two functions are equal. If the two functions are not equal, then give an element of the domain on which the two functions have different values.\\
 \begin{enumerate}[label=(\alph*)]
   \item
\[ f: \Z \to \Z, \text{ where } f(x)= x^2.\]
\[ g: \Z \to \Z, \text{ where } g(x)= |x|^2.\]\\\\
%Enter your answer below this comment line.
Both functions are equal as both the domain and target are equivalent to one another.
\\\\
\item  \[ f: \Z\times \Z \to \Z, \text{ where } f(x,y)= |x+y|.\]
\[ g: \Z\times \Z \to \Z, \text{ where } g(x,y)= |x|+|y|.\]\\\\
%Enter your answer below this comment line.
These functions are equal, though originally I thought otherwise. The reason I thought so was due to the difference between the targets, as the domains are equal. Despite this, for all integers, one instance in which this was real (as 0 is considered a valid integer) is if both x and y = 0. Another instance is in the case that both x and y are either \geq or \leq to \ 0.
\\\\
\end{enumerate}

 \newpage
%--------------------------------------------------------------------------------------------------

\section*{Problem 2}

The domain and target set of functions f and g is $\mathbb{R}$. The functions are defined as:
\begin{itemize}
  \item $f(x) = 2x + 3$\\

  \item $g(x) = 5x + 7$\\


\end{itemize}

\begin{enumerate}[label=(\alph*)]
\item  $f\circ g$?\\\\
%Enter your answer below this comment line.
$f\circ g$ = f(g(x)) \\
g(x) = 5x + 7 = f(5x+7) \\
for f = 2x+3 substitute x with 5x+7 \\
f = 2(5x+7)+3 = \\
\\
10x+17 \\

\\\\
\item  $g \circ f$?\\\\
%Enter your answer below this comment line.
$g\circ f$ = g(f(x)) \\
f(x) = 2x+3 = g(2x+3) \\
for g = 5x+7 substitute x with 2x+3 \\
5(2x+3)+7 = \\
\\
10x+22 \\
\\\\
\item  $(f\circ g)^{-1}$?\\\\
%Enter your answer below this comment line.
$f\circ g$ = f(g(x)) \\
= f(g(x))^-^1 = g(x) = 5x+7 = f(5x+7) \\
For f = 2x+3 \ substitute \ x \ with \ 5x+7 \\
2(5x+7)+3 = (10x+17)^-^1 \\
\\
= \frac{1}{10x+17} \\

\\\\
\item  $f^{-1}\circ g^{-1}$?\\\\
%Enter your answer below this comment line.
$f^{-1}\circ g^{-1}$ = (f^-^1)((g^-^1)(x)) = (f(x)^-^1)((g(x)^-^1)(x)) \\
f(x)^-^1 = (2x+3)^-^1 \ and \ g(x)^-^1 = (5x+7)^-^1 \\
(f(x)^-^1)((5x+7)^-^1) = \\
\\
\frac{5x+7}{15x+23} \\

\\
\\
\\\\
\item  $g^{-1}\circ f^{-1}$?\\\\
%Enter your answer below this comment line.
$g^{-1}\circ f^{-1}$ = (g^-^1)((f^-^1)(x)) = (g(x)^-^1)((f(x)^-^1)(x)) \\
g(x)^-^1 = (5x+7)^-^1 \ and \ f(x)^-^1 = (2x+3)^-^1 \\
(g(x)^-^1)((2x+3)^-^1) = \\
\\
\frac{2x+3}{14x+26} \\
\\
\\
\end{enumerate}
Are any of the above equal?\\\\
%Enter your answer below this comment line.
If I were to plug 0 in for all x variables in each problem, the following would be the answers I am given.
\\
\\  A: 17 \\ 

\\B: 22 \\

$C: \frac{1}{17}
\\

$D:\frac{7}{23}
\\

$E: \frac{3}{26} \\

\\
The next question I would have in order to assert whether or not these were equal functions, was whether or not the Domain and target were all the same in comparison to one another. Right off the bat, we know that the Domain is the same given that it is $\mathbb{R}$ The range though is also $\mathbb{R}$, and in order to see if these functions are equal, we have to determine whether the outcome when 0 is substituted for x is also $\mathbb{R}$ which as we can see, A, B, C, D, and E are all real numbers. Which means that these functions in each instance are all equal to one another.
\\\\


    \newpage
%--------------------------------------------------------------------------------------------------



\section*{Problem 3}

\begin{enumerate}[label=(\alph*)]
\item  Give the matrix representation for the relation depicted in the arrow diagram. Then, express the relation as a set of ordered pairs.\\\\
The arrow diagram below represents a relation.\\
\fbox{
 \includegraphics[width=2in]{M5-fig1}
}\\\\
{\color{blue}{\bf Figure 1:} \emph{An arrow diagram shows three vertices, 1, 2, and 3. An arrow from vertex 1 points to vertex 3, and another arrow from vertex 2 points to vertex 3. Two self loops are formed, one at vertex 1 and another at vertex 2. 
}
}
\\\\
%Enter your answer below this comment line.
%For more information on creating matrices in LaTeX, see this week's module resources.
Let A = {1, 2, 3} define a relation R on A: \\
R = {(1,3), (1,1), (2,3), (2,2)}
\\

\left(\begin{array}{ccc}
1 & 0 & 1 \\
0 & 1 & 1\\
0 & 0 & 0
\end{array} \right)
\\
\\

\item Draw the arrow diagram for the relation.\\
 The domain for the relation $A$ is the set $\{2,\, 5,\, 7,\, 8,\, 11\}$. For $x$, $y$ in the domain, $xAy$ if $|x-y|$ is less than $2$.
\\\\
%Enter your answer below this comment line.

%To answer this question, you may hand-draw your solution or use a program like PowerPoint or Lucidchart.
%Take a photo or screenshot, then upload your file to this project. Note that the image you submit must be legible to your instructor.
%You will see your file name appear in the file tree. Change the "YOURFILENAMEHERE" text in the includegraphics command below. It should match the name of your uploaded file.
\includegraphics[width=5in]{M5-fig4.jpeg}
\\\\
\end{enumerate}
 \newpage
%--------------------------------------------------------------------------------------------------

\section*{Problem 4}

For each relation, indicate whether the relation is:
\begin{itemize}
  \item Reflexive, anti-reflexive, or neither
  \item Symmetric, anti-symmetric, or neither
  \item Transitive or not transitive
\end{itemize}
Justify your answer.\\
\begin{enumerate}[label=(\alph*)]
\item The domain of the relation $L$ is the set of all real numbers. For $x$, $y \in \Real, \; xLy$ if $x < y$.\\\\
%Enter your answer below this comment line.
L is anti-reflexive, neither symmetric, or anti-symmetric, and transitive \\
\\
L is anti-reflexive due to the fact that there is no instance in which for all $x \in L$
it is not true that x L x. In this case, x cannot be $<$ itself in any instance which is why it is anti-reflexive.
\\
\\
The relation is not symmetric because it is true that 3 $<$ 4 but it is not true that 4 $<$ 3
\\

The relation is not anti-symmetric because 3.5 = 3.3 = 3 which means that 3.5 $<$ 3.3 $<$ 3 which is not true either. Meaning that L is neither symmetric or anti-symmetric
\\

The relation is Transitive because if x $<$ y and y $<$ z  then x $<$ y
\\\\

\item The domain of the relation $A$ is the set of all real numbers. $xAy$ if $|x-y| \leq 2$\\\\
%Enter your answer below this comment line.
A is neither reflexive or anti-reflexive, neither symmetric, or anti-symmetric, and Transitive. \\

A is neither reflexive or anti-reflexive because there are instances in which it is true that for x, \abs{x - x} \ $\leq$ 2 but there are also instances in which it is not, such as 2 = x which would show 4 $\leq$ 2
\\
The relation is symmetric because $\abs{1-1}$ $\leq$ 2
and it is true that $\abs{1-1}$ $\leq$ 2
and if and only if they both are the same outcome.
\\
\\
If \abs{x-y}$\leq2$ then that means that every value that is valid has to be transitive, because if they are not all $\leq2$ in some way, then the equation does not follow through. Though it cannot be transitive due to the fact that for the data-set in the case that it would be $\geq2$ then there would be a disconnect between those that are 
$\leq2$ so the answer is neither.
\\\\ \\
\\
\item The domain of the relation $Z$ is the set of all real numbers. $xZy$ if $y=2x$\\\\\\\\
%Enter your answer below this comment line.
2 is related to itself so it cannot be anti-reflexive, 3 is not related to itself, therefore it cannot be reflexive, meaning it is neither. \\
\\
This is not symmetric because 6 is related to 3, shown with 6 = 2(3) but 3 = 2(6) is not accurate. Which means that this is anti-symmetric \\
\\
2 is related to 4 because 4 = 2(2) but 4 is not related to 2 because 2 = 4(2) does not equate to an accurate answer. 
\\\\
\end{enumerate}
\newpage
\section*{Problem 5}

The number of watermelons in a truck are all weighed on a scale. The scale rounds the weight of every watermelon to the nearest pound. The number of pounds read off the scale for each watermelon is called its measured weight. The domain for each of the following relations below is the set of watermelons on the truck. For each relation, indicate whether the relation is:
\\
\begin{itemize}
  \item Reflexive, anti-reflexive, or neither
  \item Symmetric, anti-symmetric, or neither
  \item Transitive or not transitive
\end{itemize}
Justify your answer.\\

\begin{enumerate}[label=(\alph*)]
\item Watermelon $x$ is related to watermelon $y$ if the measured weight of watermelon $x$ is at least the measured weight of watermelon $y$. No two watermelons have the same measured weight.\\\\
%Enter your answer below this comment line.
The relation is reflexive because x is related to x for wall watermelon x. If x has a weight of z pounds then x has greater than or equal to z pounds. \\
\\
It is not symmetric because if x has 10 pounds and y has 11 pounds then y is related to x but x is not related to y. In addition to this, no watermelon has the same weight, meaning that it is anti symmetric so x = y\\
\\
The relation is transitive because if x is related to y that means that y has to be related to z because they are all interconnected through the greater than or equal to factor of the weight in addition to no watermelon being the same.
\\\\
\item Watermelon $x$ is related to watermelon $y$ if the measured weight of watermelon $x$ is at least the measured weight of watermelon $y$. All watermelons have exactly the same measured weight.\\\\
%Enter your answer below this comment line.
The relation is reflexive, Symmetric, and transitive because all watermelons have the same weight because for each iteration of x is greater than or equal y then y is also greater than or equal to x.\\
\\
\\\\
\end{enumerate}
 \newpage
%--------------------------------------------------------------------------------------------------

\section*{Problem 6}
\subsection*{Part 1}
Give the adjacency matrix for the graph G as pictured below:\\
\\
\fbox{
 \includegraphics[width=1.75in]{M5-fig2}\\
}\\\\
{\color{blue}{\bf Figure 2:} \emph{A graph shows 6 vertices and 9 edges. The vertices are 1, 2, 3, 4, 5, and 6, represented by circles. The edges between the vertices are represented by arrows, as follows: 4 to 3; 3 to 2; 2 to 1; 1 to 6; 6 to 2; 3 to 4; 4 to 5; 5 to 6; and a self loop on vertex 5.
}
}\\
%Enter your answer below this comment line.
\\
A directed graph G with 6 vertices can be represented by a 6 x 6 matrix over the set {0,1}.\\
\\
A  = \left( \begin{array}{cccccc}
0 & 0 & 0 & 0 & 0 & 1\\
1 & 0 & 0 & 0 & 0 & 0\\
0 & 1 & 0 & 0 & 0 & 0\\
0 & 0 & 0 & 0 & 1 & 0\\
0 & 0 & 0 & 0 & 1 & 1\\
0 & 0 & 0 & 0 & 1 & 0 \end{array} \right)~~~~~~~
%For more information on creating matrices in LaTeX, see this week's module resources.
\\\\

\subsection*{Part 2}
A directed graph $G$ has 5 vertices, numbered 1 through 5. The $5\times 5$ matrix $A$ is the adjacency matrix for $G$. The matrices $A^2$ and $A^3$ are given below.
\[
A^2  = \left( \begin{array}{ccccc}
0 & 1 & 0 & 0 & 0 \\
0 & 0 & 1 & 0 & 0\\
1 & 0 & 0 & 0 & 0\\
1 & 0 & 0 & 1 & 0\\
0 & 1 & 1 & 0 & 1
\end{array} \right)~~~~~~~
\]
\[
A^3  = \left( \begin{array}{ccccc}
1 & 0 & 0 & 0 & 0 \\
0 & 1 & 0 & 0 & 0\\
0 & 0 & 1 & 0 & 0\\
0 & 1 & 1 & 0 & 1\\
1 & 1 & 0 & 1 & 0
\end{array} \right)~~~~~~~
\]
Use the information given to answer the questions about the graph G.
\begin{enumerate}[label=(\alph*)]
\item Which vertices can reach vertex 2 by a walk of length 3?\\\\
%Enter your answer below this comment line.
The second, fourth, and fifth row of $A^3$ has a 1 in column 2. This means that there are three vertices that can reach vertex 2 by a walk of length 3. (2,2) (4,2) and (5,2).
\\
\\\\

\item Is there a walk of length 4 from vertex 4 to vertex 5 in $G$? (Hint: $A^4 = A^2\cdot A^2$.)\\\\
%Enter your answer below this comment line.
\\
A^4  = \left( \begin{array}{ccccc}
0 & 0 & 1 & 0 & 0 \\
1 & 0 & 0 & 0 & 0\\
0 & 1 & 0 & 0 & 0\\
1 & 1 & 0 & 1 & 0\\
1 & 1 & 2 & 0 & 1
\end{array} \right)~~~~~~~
\end{enumerate}
\\
There are in fact, walks of length 4 in G that start at vertex 4, and end at vertices 1, 2, and 3. To be more specific, the vertices are the following. (4,1), (4,2), (4,3).
 \newpage
%--------------------------------------------------------------------------------------------------

\section*{Problem 7}
\subsection*{Part 1} The drawing below shows a Hasse diagram for a partial order on the set $\{A,\, B,\, C,\, D,\, E,\, F,\, G,\, H,\, I,\, J\}$
\\
\\
\fbox{
 \includegraphics[width=2in]{M5-fig3}\\
}\\
\\
{\color{blue}{\bf Figure 3:} \emph{A Hasse diagram shows 10 vertices and 8 edges. The vertices, represented by dots, are as follows: vertex J; vertices H and I are aligned vertically to the right of vertex J; vertices A, B, C, D, and E forms a closed loop, which is to the right of vertices H and I; vertex G is inclined upward to the right of vertex E; and vertex F is inclined downward to the right of vertex E. The edges, represented by line segments, between the vertices are as follows: Vertex J is connected to no vertex; a vertical edge connects vertices H and I; a vertical edge connects vertices B and C; and 6 inclined edges connect the following vertices, A and B, C and D, D and E, A and E, E and G, and E and F.
}
}
\\
\\

\begin{enumerate}[label=(\alph*)]
\item What are the minimal elements of the partial order?\\\\
%Enter your answer below this comment line.
The minimal elements are A, I, and J.
\\\\

\item What are the maximal elements of the partial order?\\\\
%Enter your answer below this comment line.
The maximal elements are J, H, G, and D
\\\\

\item Which of the following pairs are comparable?\\
\[
(A, \,D),\, (J,\, F),\, (B,\, E),\, (G, \,F),\, (D,\, B),\, (C, \,F),\, (H,\, I), \,(C,\, E)
\]\\\\
%Enter your answer below this comment line.
(A,D), (G,F), (D,B), (H,I) are all comparable. The reason they are comparable is due to the fact that all are easily accessible to each other without going downwards after the initial movement from bottom to top.
\\\\

\end{enumerate}

\subsection*{Part 2}
Each relation given below is a partial order. Draw the Hasse diagram for the partial order.

\begin{enumerate}[label=(\alph*)]
\item The domain is $\{3,\, 5,\, 6,\, 7,\, 10,\, 14,\, 20,\, 30,\, 60\}$. $x \leq y$ if $x$ evenly divides $y$.\\\\
%Enter your answer below this comment line.
\item The domain is $\{a,\, b,\, c,\, d,\, e,\, f\}$. The relation is the set:
\[
\{ (b,\, e),\, (b,\, d),\, (c,\, a),\, (c,\, f),\, (a,\, f),\, (a,\, a),\, (b,\, b),\, (c, \,c),\, (d,\, d),\, (e, \,e), \,(f,\, f) \}
\]\\\\
%To answer this question, you may hand-draw your solution or use a program like PowerPoint or Lucidchart.
%Take a photo or screenshot, then upload your file to this project. Note that the image you submit must be legible to your instructor.
%You will see your file name appear in the file tree. Change the "YOURFILENAMEHERE" text in the includegraphics command below. It should match the name of your uploaded file.
\includegraphics[height=6in, width=6in]{M5-fig5.jpeg}
\\\\
%Enter your answer below this comment line.
\\\\

\end{enumerate}

\newpage%--------------------------------------------------------------------------------------------------

\section*{Problem 8}

Determine whether each relation is an equivalence relation. Justify your answer. If the relation is an equivalence relation, then describe the partition defined by the equivalence classes.\\
\begin{enumerate}[label=(\alph*)]
\item The domain is a group of people. Person $x$ is related to person $y$ under relation $M$ if $x$ and $y$ have the same favorite color. You can assume that there is at least one pair in the group, $x$ and $y$, such that $xMy$.\\\\
%Enter your answer below this comment line.
The relation is a great example of an equivalence relation. Let us say that the group of people only consists of person x and person y. In this case, the relation M is the only relation, and if person x and person y both have the same favorite color, that means that the relation is reflexive, since everyone has the same favorite color, symmetric, because everyone has the same favorite color, and transitive, because everyone has the same favorite color.\\
\\
x\sim y, y\sim x, x\sim x, y\sim y
\\\\

\item The domain is the set of all integers. $xEy$ if $x + y$ is even. An integer $z$ is even if $z = 2k$ for some integer $k$.\\\\
%Enter your answer below this comment line.
If x + y is even, is y + x even?\\
For x = 1 and y = 1\\
The answer to this is yes meaning that this relation is reflexive\\
\\
Is x + x even, for all x?\\
For all x, x + x is in fact even. Showing that the relation is also symmetric.\\
\\
If x + y is even, and y + z is even, is x + z even?\\
for x = 1, y = 1, and z = 1\\
if 1 + 1 = 2 is even, and 1 + 1 is even, is 1 + 1 even? The answer is yes, meaning that we can prove that this relation is also transitive meaning that it is in fact an equivalence relation.\\
\\
x\sim x, x\sim y, x\sim z, y\sim z, y\sim y, z\sim z, y\sim x, z\sim x, z\sim y 
\\\\

\end{enumerate}
%--------------------------------------------------------------------------------------------------



\end{document}
